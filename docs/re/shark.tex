\documentclass[a4paper,oneside,12pt]{book}
\usepackage[T1]{fontenc}
\usepackage{CJKutf8}
\usepackage{color}
\usepackage{graphicx}
\usepackage{mathptmx}
\newenvironment{SChinese}{
  \CJKfamily{gkai}
  \CJKtilde
  \CJKnospace}{}
\begin{document}
%~ \renewcommand{\rmdefault}{ptm}
\begin{CJK*}{UTF8}{kai}
\begin{SChinese}
\begin{center}
\huge{知识库系统需求说明}\\
 \textcolor{blue}{\Large{知识系统的随笔博客实现}\\}
 \normalsize{}
\vspace{1cm} 
\begin{tabular}{|l|l|}
\hline
开发者\\
\hline
赵林\\
\hline
解朝勇\\
\hline
\end{tabular}
\end{center}
\begin{flushleft}
%\hspace{3cm} 图二 运行结果\\
\section*{1.模糊需求}
\begin{itemize}
\item{写blog}
\item{email 定时推送, 订阅的形式,将排名前几个的blog发送到用户邮箱}
\item{用户管理,注册,登陆, 注册的时候,写验证码, 使用reCapture}
\item{用户配置网站皮肤}
\item{博客内容允许使用html, 禁用js}
\item{博客采用模板,方便样式切换}
\item{允许用户上传图片,后台提供图片处理模块}
\item{防止机器人灌水,验证码}
\item{防止恶意访问,洪水攻击,ip频繁访问,要求输入验证码,不再响应需求}
\item{对于加载很慢的资源,进行缓存,缓存设置一个阈值,防止缓存过大,内存泄露}
\item{插件机制, 通过hook来扩展系统}
\item{I18N, 多语言版本}
\item{在网站根目录下建立docs文件夹,保存基本的设计文档}
\item{将HTML和文本分离,提供打印版本,打印博客, 则去除一些图片,只打印文本}
\end{itemize}
\section*{2.整理需求}
\begin{enumerate}
\item{ Blog编写与发布,博客内容允许使用html,禁用js,可上传图片}
\item{ Blog订阅(通过定时email)}
\item{ 用户管理,注册,登陆, 注册(需要验证码)}
\item{ 用户配置网站皮肤}
\item{ Blog评论}
\item{ 互粉}
\item{ 安全性,如sql注入,ip黑名单}
\item{ 访问速度}
\item{ 系统架构要方便功能扩展}
\item{ 支持多国语言}
\item{ Blog导出功能}
\end{enumerate}
\end{flushleft}
\end{SChinese}
\end{CJK*}
\end{document}
